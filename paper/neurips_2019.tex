\documentclass{article}

% if you need to pass options to natbib, use, e.g.:
%     \PassOptionsToPackage{numbers, compress}{natbib}
% before loading neurips_2019

% ready for submission
% \usepackage{neurips_2019}

% to compile a preprint version, e.g., for submission to arXiv, add add the
% [preprint] option:
%     \usepackage[preprint]{neurips_2019}

% to compile a camera-ready version, add the [final] option, e.g.:
     \usepackage[final]{neurips_2019}

% to avoid loading the natbib package, add option nonatbib:
%     \usepackage[nonatbib]{neurips_2019}

\usepackage[utf8]{inputenc} % allow utf-8 input
\usepackage[T1]{fontenc}    % use 8-bit T1 fonts
\usepackage{hyperref}       % hyperlinks
\usepackage{url}            % simple URL typesetting
\usepackage{booktabs}       % professional-quality tables
\usepackage{amsfonts}       % blackboard math symbols
\usepackage{nicefrac}       % compact symbols for 1/2, etc.
\usepackage{microtype}      % microtypography

\title{Adversarial Attacks on Capsule Networks}
\bibliographystyle{plain}
% The \author macro works with any number of authors. There are two commands
% used to separate the names and addresses of multiple authors: \And and \AND.
%
% Using \And between authors leaves it to LaTeX to determine where to break the
% lines. Using \AND forces a line break at that point. So, if LaTeX puts 3 of 4
% authors names on the first line, and the last on the second line, try using
% \AND instead of \And before the third author name.



\author{%
  Felix Michels, Tobias Uelwer, Stefan Harmeling \\
  Department of Computer Science\\
  Heinrich-Heine University Düsseldorf\\
  \texttt{\{felix.michels, tobias.uelwer, harmeling\}@hhu.de} \\
  % examples of more authors
  % \And
  % Coauthor \\
  % Affiliation \\
  % Address \\
  % \texttt{email} \\
  % \AND
  % Coauthor \\
  % Affiliation \\
  % Address \\
  % \texttt{email} \\
  % \And
  % Coauthor \\
  % Affiliation \\
  % Address \\
  % \texttt{email} \\
  % \And
  % Coauthor \\
  % Affiliation \\
  % Address \\
  % \texttt{email} \\
}
\begin{document}

\maketitle

\begin{abstract}
	In this paper we want to extensively evaluate the robustness of capsule networks towards different adversarial attacks. Our experiments show that capsule networks can be fooled as easily as convolutional networks.

\end{abstract}

\section{Introduction}



\section{Capsule Networks and Dynamic Routing}

\cite{capsules}

\section{Adversarial Attacks}

\subsection{Carlini-Wagner Attack}

\cite{carlini}

\subsection{Boundary Attack}

\cite{boundary}

\subsection{DeepFool Attack}

\cite{deepfool}

\subsection{Universal Adversarial Perturbations}
\cite{universal}

\section{Experiments}

\subsection{Capsule Network}

\subsection{Datasets}

We train the capsule network on the following benchmark datasets:
\begin{itemize}
	\item MNIST \cite{mnist}
	\item Fashion-MNIST \cite{fashion}
	\item SVHN \cite{svhn}
	\item CIFAR10 \cite{cifar}
\end{itemize}

\section{Conclusion}


\bibliography{neurips_2019}


\end{document}
