%%% Die folgende Zeile nicht ändern!
\section*{\ifthenelse{\equal{\sprache}{deutsch}}{Zusammenfassung}{Abstract}}
%%% Zusammenfassung:

This bachelor thesis extensively evaluates the vulnerability of capsule network to various adversarial attacks.
Capsule networks are a form of neural network in which neurons are grouped in capsules whose activity vectors describe different properties of the same entity.
Recent work attributes them more robustness to adversarial attacks than other types of neural networks.

To investigate those conjectures we train four pairs of capsule networks and convolutional neural networks discriminatively on different image datasets.
Using multiple strong and diverse adversarial attack we compute adversarial examples for each combination of dataset and architecture.

By comparing the euclidean norms of the generated perturbations we find that capsule networks are vulnerable to adversarial attacks in a similar level as convolutional networks.
We discover in particular that capsule networks show no robustness to sufficiently strong white-box attacks.
Additionally we demonstrate limited transferability of the adversarial examples between the two types of architectures.

We study furthermore the structure of the adversarial perturbations and can observe that they seem to lie in a relatively low dimensional linear subspace in both cases.
Despite those similarities, using dimensionality reduction we detect inherent differences between the perturbations constructed for the different network types.

Finally we aim to identify the source of adversarial examples in the networks we trained.
Since capsule networks and convolutional partially use the same of convolutional layer we focus on the disturbance on the activation of individual layers caused by adversarial attacks.
We find that susceptibility to attacks is not just ascribable to the first layers in a network but a result of deeper layers as well.
Specifically we find that the dynamic routing algorithm used in capsule networks does not establish capsule activation invariants to counteract adversarial attacks, as previously conjectured.