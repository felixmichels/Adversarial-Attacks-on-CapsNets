% !TeX root = ba.tex

\section{Capsule Networks}

\todo{Rewrite this whole section}

The concept of vector capsules and the dynamic routing algorithm was proposed by \citet{capsules}. In an essence, neurons are grouped into vectors, so-called capsules. Each capsule vector is dedicated to a distinct abstract entity, i.e. a single object class in a classification setting. The norm of a capsule vector encodes the probability of the represented object being present in the input, while the vector orientation encodes the object's characteristics. Thus, CapsNets aim to develop dedicated representations that are distributed into multiple vectors in contrast to convolutional networks that utilize an entangled representation in a single vector at a given location. This allows the application of linear transformations directly to the representations of respective entities. Spatial relations, which can be implemented as a matrix product, can thus be modeled more efficiently.

CapsNets are organized in layers. Initially, the original CapsNet applies a convolutional layer. The resulting feature maps are then processed by the primary capsule layer. Internally, it applies a series of convolutional layers on its own, each yielding a spatial grid of capsules. Within all capsule layers the \emph{squashing} function serves as a vector-to-vector non-linearity that squashes each capsule vector length between $0$ and $1$ while leaving the orientation unaltered. Subsequently, convolutional or densely connected capsule layers can be applied. While the latter does not utilize weight sharing, convolutional capsule layers share the kernels over the spatial grid, as well as capsules from the previous layer. These layers effectively estimate output capsules based on respective input capsules. The dynamic routing algorithm determines each agreement between estimate and iteratively calculated output capsule. This is done by introducing a scalar factor, the so-called routing coefficient, for each connection between an estimate and respective output. Such an output is defined as the sum over all respective estimates, weighted by their routing coefficients. Theoretically, that means information flows where it is needed, both during forward and backpropagation. This non-parametric procedure supports the goal of capsules with clean dedicated representations. To improve results, an additional capsule may be used within the routing algorithm to serve as a dead end for information that may not be linked to known abstract capsule categories. This is also referred to as the \emph{none-of-the-above} category.

- History of capsule networks (Hinton, Sabour) \\
- Explain Dynamic Routing \\
- Mention other routing algorithms \\
- Parse tree \\
- Representation??? \\

Squashing:
\begin{equation}
\begin{aligned}
& v_j = \frac{\norm{s_j}}{1 + \norm{s_j}^2} s_j \\
& \sum_{i} c_{ij} \hat{u}_{j \vert i}, & & \hat{u}_{j \vert i} = W_{ij}u_j
\end{aligned}
\end{equation}