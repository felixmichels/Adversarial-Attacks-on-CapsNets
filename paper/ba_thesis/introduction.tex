\section{Introduction}
\label{sec:introduction}

Capsule networks (CapsNets) are a novel form of artificial neural networks introduced by \citet{capsules,em} which partition the neurons of a layer into groups called \emph{capsules} and route their output dynamically.
This aims to preserve information about hierarchical spatial relationships between objects and thereby to generalize better to new viewpoints.
In some computer vision tasks CapsNets have reached the state-of-the-art of convolutional neural networks (ConvNets), and have surpassed them in some particular functions like segmenting highly overlapping objects.
As a possible alternative to ConvNets, they also face some of the same challenges.

With the recent emergence of deep learning in many real-world applications, from face recognition to brain tumor detection and satellite imagery analysis, security in machine learning causes growing concerns.
Adversarial attacks describe any technique designed to cause machine learning models to perform poorly.
An attack can start before the model is even trained, e.g.\ by maliciously manipulating the training data (poisoning), but test-time attacks are more widely applicable and pose a bigger threat.

In computer vision tasks specifically \emph{adversarial examples} \citep{intriguing} refer to images  that are designed in such a way that a machine learning system make mistakes, but still appear to be an ordinary image to a human viewer.
\citet{fgsm} hypothesize that linear behavior in high-dimensional spaces is the causes for the existence of adversarial examples and develop an attack which finds adversarial examples for deep neural networks with particular ease.
This is not just problematic for security-sensitive applications like self-driving cars, but also casts doubt an their ability to generalize in a meaningful way.

The effects of adversarial attacks on CapsNets have not yet been studied in a satisfactory manner.
We will investigate the robustness of CapsNets in comparison to ConvNets against multiple common adversarial attacks and attack scenarios across four different popular image data sets.
We also look at the structural difference of the generated adversarial examples, the transferability of perturbations between architectures and unit activations between layers.
We show that CapsNets are in general no less susceptible to adversarial examples than ConvNets.

This thesis is structured as follows: In \Cref{sec:capsules} we reiterate the concept of capsule networks and the routing-by-agreement algorithm. In \Cref{sec:attacks} we describe the attacks we use to test the CapsNets. We present the results and interpretations of our experiments in \Cref{sec:experiments}.

\section{Related Work}
\label{sec:related}

\citet{em} show that CapsNets with EM Routing are less vulnerable to the simple white-box attacks FGSM \citep{fgsm} and the Basic Iterative Method \citep{bim} than standard ConvNets.
\citet{capsattacktraffic} study adversarial robustness of CapsNets using the usual routing-by-agreement, although they focus on the performance of their own proposed black-box algorithm and their own definition of imperceptibility.
\citet{darccc} claim the robustness of CapsNets against white-box attacks and approach the problem further by detecting adversarial examples through use of the reconstruction network which is trained alongside the CapsNet in most applications anyway.
\citet{scaledagreement} introduce a further routing algorithm and demonstrate its robustness against FGSM.