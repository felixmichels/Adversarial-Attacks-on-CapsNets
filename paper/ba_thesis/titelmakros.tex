

%%%%%%%%%%%%%%%%%%%%%%%%%%%%%%%%%%%%%%%%%%%%%%%%%%%%%%%%%%%
% Obere Titelmakros. Editieren Sie diese Datei nur, wenn
% Sie sich ABSOLUT sicher sind, was Sie da tun!!!
% (Z.B. zum Abaendern der BA-Vorlage in eine MA-Vorlage)
% Uni Duesseldorf
% Lehrstuhl fuer Datenbanken und Informationssysteme
% Version 2.2 - 2.3.2010
%%%%%%%%%%%%%%%%%%%%%%%%%%%%%%%%%%%%%%%%%%%%%%%%%%%%%%%%%%%
\newcommand{\AN}{twoside}
\newcommand{\AUS}{}
%\newcommand{\englisch}{}
%\newcommand{\deutsch}{\usepackage[german]{babel}}

%% Die folgenden auskommentierten Optionen dienen der automatischen
%% Erkennung des Latex-Kompilers und dem Setzen der davon abhängigen
%% Einstellungen. Bei Problem z.B. mit dem Einbinden von verschiedenen
%% Grafiktypen bei Verwendung von PdfLatex oder Latex, einfach die
%% verschiedenen \usepackage(s) ausprobieren. (Mit diesen Einstellungen
%% funktionierte diese Vorlage bei der Verwenundg von latex.exe als
%% Kompiler bei den meisten Studierenden.)

%\newif\ifpdf \ifx\pdfoutput\undefined
%\pdffalse % we are not running pdflatex
%\else
%\pdfoutput=1 % we are running pdflatex
%\pdfcompresslevel=9 % compression level for text and image;
%\pdftrue \fi

\documentclass[11pt,a4paper, \zweiseitig]{article}



%\usepackage[iso]{umlaute}
\usepackage[utf8]{inputenc}
\usepackage{palatino} % palatino Schriftart
%\usepackage{makeidx} % um ein Index zu erstellen
\usepackage[nottoc]{tocbibind}
\usepackage[T1]{fontenc} %fuer richtige Trennung bei Umlauten
\usepackage{fancybox} % fuer die Rahmen
\usepackage{shortvrb}
\usepackage{url}
\usepackage{xcolor}
\usepackage[colorlinks,citecolor=blue,linkcolor=black]{hyperref} %anklickbares Inhaltsverzeichnis
\usepackage[style=authoryear,natbib=true,backend=biber,mincitenames=1,maxcitenames=2,maxbibnames=99,uniquelist=false,dashed=false]{biblatex}

% https://tex.stackexchange.com/a/334703/8850
\AtEveryBibitem{%
  \clearfield{issn}
  \clearfield{isbn}
  \clearfield{doi}
  \clearfield{location}
  \clearlist{location}
  \clearlist{address}

  \ifentrytype{online}{}{% Remove url except for @online
    \clearfield{url}
  }
}

% Falls es bei \citet ein Komma zwischen Name und Jahr gibt:
% https://tex.stackexchange.com/questions/312539/unwanted-comma-between-author-and-year-using-citet-command
% (thx @ Markus Brenneis)
%\DeclareDelimFormat[cbx@textcite]{nameyeardelim}{\addspace}



\usepackage{ifthen}
\ifthenelse{\equal{\sprache}{deutsch}}{
 \usepackage[ngerman]{babel}
 % Bibtex u.a -> et al.
 \DefineBibliographyStrings{ngerman}{
   andothers = {{et\,al\adddot}},
 }
 \usepackage[autostyle, german=quotes]{csquotes} % Deutsche Anführungszeichen im Literaturverzeichnis (thx @ Markus Brenneis)
 }{}

\usepackage{a4wide} % ganze A4 Weite verwenden



%\ifpdf
%\usepackage[pdftex,xdvi]{graphicx}
%\usepackage{thumbpdf} %thumbs fuer Pdf
%\usepackage[pdfstartview=FitV]{hyperref} %anklickbares Inhaltsverzeichnis
%\else
%\usepackage[dvips,xdvi]{graphicx}
\usepackage{graphicx}

%\fi

\newcommand{\redt}[1] {
  \textcolor{red}{#1}}

\newcommand{\oranget}[1] {
  \textcolor{orange}{#1}}

\newcommand{\purplet}[1] {
  \textcolor{purple}{#1}}

%%%%%%%%%%%%%%%%%%%%%%% Massangaben fuer die Arbeit %%%%%%%%%%%%%%%
\setlength{\textwidth}{15cm}

\setlength{\oddsidemargin}{35mm}
\setlength{\evensidemargin}{25mm}

\addtolength{\oddsidemargin}{-1in}
\addtolength{\evensidemargin}{-1in}

\addbibresource{references.bib}

%\makeindex

\begin{document}

%\setcounter{secnumdepth}{4} %Nummerieren bis in die 4. Ebene
%\setcounter{tocdepth}{4} %Inhaltsverzeichnis bis zur 4. Ebene

\pagestyle{headings}

\sloppy % LaTeX ist dann nicht so streng mit der Silbentrennung
%~ \MakeShortVerb{\§}

\parindent0mm
\parskip0.5em


{
\textwidth170mm 
\oddsidemargin30mm 
\evensidemargin30mm 
\addtolength{\oddsidemargin}{-1in}
\addtolength{\evensidemargin}{-1in}

\parskip0pt plus2pt

% Die Raender muessen eventuell fuer jeden Drucker individuell eingestellt
% werden. Dazu sind die Werte fuer die Abstaende `\oben' und `\links' zu
% aendern, die von mir auf jeweils 0mm eingestellt wurden.

%\newlength{\links} \setlength{\links}{10mm}  % hier abzuaendern
%\addtolength{\oddsidemargin}{\links}
%\addtolength{\evensidemargin}{\links}

\begin{titlepage}
\vspace*{-1.5cm}
  \raisebox{17mm}{
    \begin{minipage}[t]{70mm}
      \begin{center}
        %\selectlanguage{german}
        {\Large INSTITUT FÜR INFORMATIK\\}
        {\normalsize
          Datenbanken und Informationssysteme\\
        }
        \vspace{3mm}
        {\small Universitätsstr. 1 \hspace{5ex} D--40225 Düsseldorf\\}
     \end{center}
    \end{minipage}
  }
  \hfill
  \includegraphics[width=130pt]{bilder/HHU_Logo}
  \vspace{14em}

% Titel
  \begin{center}
      	\baselineskip=55pt
    	\textbf{\huge \titel}
  	 	\baselineskip=0 pt
   \end{center}

  %\vspace{7em}

\vfill

% Autor
  \begin{center}
    \textbf{\Large
      \bearbeiter
    }
  \end{center}

  \vspace{35mm}
 
% Prüfungsordnungs-Angaben
  \begin{center}
    %\selectlanguage{german}
    
%%%%%%%%%%%%%%%%%%%%%%%%%%%%%%%%%%%%%%%%%%%%%%%%%%%%%%%%%%%%%%%%%%%%%%%%%
% Ja, richtig, hier kann die BA-Vorlage zur MA-Vorlage gemacht werden...
% (nicht mehr nötig!)
%%%%%%%%%%%%%%%%%%%%%%%%%%%%%%%%%%%%%%%%%%%%%%%%%%%%%%%%%%%%%%%%%%%%%%%%%
    {\Large \arbeit}

    \vspace{2em}

    \begin{tabular}[t]{ll}
      Beginn der Arbeit:& \beginndatum \\
      Abgabe der Arbeit:& \abgabedatum \\
      Gutachter:         & \erstgutachter \\
                         & \zweitgutachter \\
    \end{tabular}
  \end{center}

\end{titlepage}

}

%%%%%%%%%%%%%%%%%%%%%%%%%%%%%%%%%%%%%%%%%%%%%%%%%%%%%%%%%%%%%%%%%%%%%
\clearpage
\begin{titlepage}
  ~                % eine leere Seite hinter dem Deckblatt
\end{titlepage}
%%%%%%%%%%%%%%%%%%%%%%%%%%%%%%%%%%%%%%%%%%%%%%%%%%%%%%%%%%%%%%%%%%%%%
\clearpage
\begin{titlepage}
\vspace*{\fill}

\section*{Erklärung}

%%%%%%%%%%%%%%%%%%%%%%%%%%%%%%%%%%%%%%%%%%%%%%%%%%%%%%%%%%%
% Und hier ebenfalls ggf. BA durch MA ersetzen...
% (Auch nicht mehr nötig!)
%%%%%%%%%%%%%%%%%%%%%%%%%%%%%%%%%%%%%%%%%%%%%%%%%%%%%%%%%%%

Hiermit versichere ich, dass ich diese \arbeit{}
selbstständig verfasst habe. Ich habe dazu keine anderen als die
angegebenen Quellen und Hilfsmittel verwendet.

\vspace{25 mm}

\begin{tabular}{lc}
Düsseldorf, den \abgabedatum \hspace*{2cm} & \underline{\hspace{6cm}}\\
& \bearbeiter
\end{tabular}

\vspace*{\fill}
\end{titlepage}

%%%%%%%%%%%%%%%%%%%%%%%%%%%%%%%%%%%%%%%%%%%%%%%%%%%%%%%%%%%%%%%%%%%%%
% Leerseite bei zweiseitigem Druck
%%%%%%%%%%%%%%%%%%%%%%%%%%%%%%%%%%%%%%%%%%%%%%%%%%%%%%%%%%%%%%%%%%%%%

\ifthenelse{\equal{\zweiseitig}{twoside}}{\clearpage\begin{titlepage}
~\end{titlepage}}{}

%%%%%%%%%%%%%%%%%%%%%%%%%%%%%%%%%%%%%%%%%%%%%%%%%%%%%%%%%%%%%%%%%%%%%
\clearpage
\begin{titlepage}

%%% Die folgende Zeile nicht ändern!
\section*{\ifthenelse{\equal{\sprache}{deutsch}}{Zusammenfassung}{Abstract}}
%%% Zusammenfassung:
Hier kommt eine ca.\ einseitige Zusammenfassung der Arbeit rein.



%%%%%%%%%%%%%%%%%%%%%%%%%%%%%%%%%%%%%%%%%%%%%%%%
% Untere Titelmakros. Editieren Sie diese Datei nur, wenn Sie sich
% ABSOLUT sicher sind, was Sie da tun!!!
%%%%%%%%%%%%%%%%%%%%%%%%%%%%%%%%%%%%%%%%%%%%%%%
\vspace*{\fill}
\end{titlepage}

%%%%%%%%%%%%%%%%%%%%%%%%%%%%%%%%%%%%%%%%%%%%%%%%%%%%%%%%%%%%%%%%%%%%%
% Leerseite bei zweiseitigem Druck
%%%%%%%%%%%%%%%%%%%%%%%%%%%%%%%%%%%%%%%%%%%%%%%%%%%%%%%%%%%%%%%%%%%%%
\ifthenelse{\equal{\zweiseitig}{twoside}}
  {\clearpage\begin{titlepage}~\end{titlepage}}{}
%%%%%%%%%%%%%%%%%%%%%%%%%%%%%%%%%%%%%%%%%%%%%%%%%%%%%%%%%%%%%%%%%%%%%
\clearpage \setcounter{page}{1}
\pagenumbering{roman}
\setcounter{tocdepth}{2}
\tableofcontents

%\enlargethispage{\baselineskip}
\clearpage
%%%%%%%%%%%%%%%%%%%%%%%%%%%%%%%%%%%%%%%%%%%%%%%%%%%%%%%%%%%%%%%%%%%%%
% Leere Seite, falls Inhaltsverzeichnis mit ungerader Seitenzahl und 
% doppelseitiger Druck
%%%%%%%%%%%%%%%%%%%%%%%%%%%%%%%%%%%%%%%%%%%%%%%%%%%%%%%%%%%%%%%%%%%%%
\ifthenelse{ \( \equal{\zweiseitig}{twoside} \and \not \isodd{\value{page}} \)}
	{\pagebreak \thispagestyle{empty} \cleardoublepage}{\clearpage}


% Kapitel soll bei doppelseitigem Druck immer auf der rechten (ungeraden) Seite anfangen (thx @ Philipp Grawe)
% https://tex.stackexchange.com/a/223387
\ifthenelse{\( \equal{\sectionforcestartright}{twoside} \)}
 {\let\oldsection\section % Store \section in \oldsection
   \renewcommand{\section}{\cleardoublepage\oldsection}}
 {}